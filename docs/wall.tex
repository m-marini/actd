\documentclass[a4paper,11pt]{article}
\usepackage[T1]{fontenc}
\usepackage[utf8]{inputenc}
\usepackage{lmodern}

\title{Neural network notes}
\author{Marco Marini}

\begin{document}

\maketitle
\tableofcontents

\begin{abstract}
Studio del gioco wall
\end{abstract}

\section{Generale}

Wall è un gioco dove una pallina si muove in un campo rettangolare con traiettorie rettilinee diagonali.
I limiti superiore e laterali sono costituiti da muri che
fanno rimbalzare la pallina.
La parte inferiore invece è aperta e una racchetta controllata
da giocatore si muove orizzontalmente permettendo allo stesso di far rimbalzare la pallina all'interno del campo da 
gioco.

$
\begin{array}{ccccccccccccccc}
=	& = & = & = & = & = & = & = & = & = & = & = & = & = & = \\
|	&  &  &  &  &  &  &  &  &  &  &  &  &  & | \\ 
|	&  &  &  &  &  &  &  &  &  &  &  &  &  & | \\ 
|	&  &  &  &  &  &  &  &  &  &  &  &  &  & | \\ 
|	&  &  &  &  &  &  &  &  &  &  &  &  &  & | \\ 
|	&  &  &  &  &  &  &  &  &  &  &  &  &  & | \\ 
|	&  &  &  &  &  & O &  &  &  &  &  &  &  & | \\ 
|	&  &  &  &  &  &  &  &  &  &  &  &  &  & | \\ 
|	&  &  &  &  &  &  &  &  &  &  &  &  &  & | \\ 
|	&  &  &  &  &  &  &  &  &  &  &  &  &  & | \\ 
	&  &  & = & = & = &  &  &  &  &  &  &  &  & 
\end{array} 
$

Definiamo 
\[ n = 10 \] il numero di righe del campo
\[ m = 13 \] il numero di colonne
\[ w = 3 \] la larghezza della racchetta


\section{Spazio degli stati}

In un qualsiasi momento lo stato del gioco è rappresentato dalla posizione
della pallina, la direzione di spostamento della pallina e la posizione della racchetta.
Calcoliamo il numero di stati possibili:

La racchetta può trovarsi in uno degli
\[m - w + 1 = 11 \] possibili posizioni.

Quando la pallina non si trova in prossimtà dei muri o della racchetta
può muoversi in 4 diverse direzioni: NE, SE, SO, NO.
Quindi abbiamo
\[ 4 (n-2)(m-2) (m - w + 1) = 3872 \]
possibili stati della pallina.

Negli angoli superiori la pallina può avere solo una direzione quindi
si aggiungono altri 
\[
	2 (m -w +1) = 22
\] stati.

Quando si trova in prossimità invece del muro superiore o di quelli laterali, la pallina può assumere solo due possibili velocità quindi avremo:
\[ (m -w + 1) 2 [ 2 (n - 2) + m - 2] = 594 \]
ulteriori stati.

Vediamo ora alcuni particolari quando la pallina si trova su nell'angolo
inferiore sx nel qual caso è possibile una sola traiettoria verso l'alto
se la racchetta si trova in prima o seconda posizione (rimbalzo) o verso 
l'uscita del campo negli altri casi, quindi avremo
\[ m-w+1 = 11 \] possibili casi.

Altrettanti se consideriamo quando la pallina si trova nell'angolo inferiore dx.

Nel caso invece la pallina si trovi sulla riga inferiore del campo avremo due possibili traiettorie: NE o  NO se la pallina si trova esattamente sotto la racchetta (rimbalzo) o SE o SO negli altri casi.

Quindi avremo
\[ 2 * (m - 2) = 22 \] stati.

Poi avremo lo stato finale di pallina fuori campo.

In totale quindi possiamo contare
\[
	3872 + 22 + 594 + 11 + 11 + 22 + 1 = 4533
\] possibili stati.


\section{Miglior policy}

La policy migliore è quella di far rimbalzare la pallina una volta che raggiunge la parte inferiore del campo

La pallina rimbalza ogni
\[
	2 (n - 1)
\]
passi.

Il ritorno aspettato allo stato dopo $ i = (1 \dots 2(n-1) $ passi dal rimbalzo sarà quindi
\[
	R_i = \gamma ^ {2(n-1)-i} \sum_{j=0}^{\infty} R^+ \gamma ^ {2(n-1)j} =
	 R^+ \frac{ \gamma ^ {2(n-1)-i}}{1-\gamma ^ {2(n-1)}}
\]


\section{Rimbalzi certi}

Contiamo ora quanti stati che generano premi positivi indipendentemente dalla strategia.

La racchetta si trova completamente a sx, la pallina può trovarsi completamente a sx con solo la direzione NE possibile, oppure si può trovare nelle due colonne successive con direzioni NO o NE, oppure in quarta colonna con direzione NO
\[ 1 + 2(w-1) + 1 = 2w =  6 \]
possibili stati.

Simmetricamente abbiamo altri 6 stati quando la racchetta si trova completamente a dx.

La racchetta si trova tra la seconda colonna e la quartultima colonna, la pallina può trovarsi nelle tre colonne successive con direzioni NO o NE
\[
2w(m-w-2) = 48
\]
possibili stati.

In tutto possiamo contare 
\[
4w+2w(m-w-2) = 2w(m-w)=60
\]
possibili stati di rimbalzo certo.


\section{Fine gioco}

Contiamo ora quanti stati dove, indipendentemente dalla strategia, si arriva a fine gioco.

Quando la racchetta si trova in prima colonna e la pallina si trova in quarta o quinta colonna con direzione SE o dalla sesta alla penultima colonna con due direzioni possibili o infine la pallina si trova in ultima colonna direzione SO
\[
2(m-w-3)+3=2(m-w)-3=17
\]
stati possibili.

Quando la racchetta si trova in seconda colonna e la pallina si trova in quinta o sesta colonna con direzione SE oppure tra la settima e la penultima colonna con due direzioni possibili o infine la pallina si trova in ultima colonna con direzione SO
\[
2(m-w-4)+3 = 2(m-w)-5 = 15
\]
stati possibili.

Quando la racchetta si trova in terza colonna e la pallina si trova in seconda colonna con direzione SO o in sesta o settima colonna con direzione SE oppure tra la ottava e la penultima colonna con due direzioni possibili o infine la pallina si trova in ultima colonna con direzione SO
\[
2(m-w-5)+4= 2(m-w)-6 = 14
\]
stati possibili.

Gli stessi stati li abbiamo per simmetria quando la racchetta si trova al lato opposto.

Quando la racchetta si trova tra la quarta colonna e sei colonne prima dell'ultima e la pallina si trova nei bordi con singole direzioni, o nelle due adiacenti la sx o la dx della racchetta con singole direzioni o nelle colonne intermedie tra i bordi e le precedenti colonne con due direzioni
\[
2(m-w-6)+6 = 2(m-w)-6=14
\]

In tutto quindi ci sono
\[
\begin{array}{r}
 2[2(m-w)-3+2(m-w)-5+2(m-w)-6]+2(m-w)-6 = \\
 = 12(m-w)-28+2(m-w)-6 = \\
 = 14(m-w)-34 = 106 \\
\end{array}
\] stati di fine certa del gioco.

\section{Rimbalzi condizionali}

Contiamo ora gli stati di rimbalzo dipendenti dalla strategia.

Quando la racchetta si trova in prima colonna e la pallina in quinta colonna con direzione SO 
\[
1
\]
stato possibile.

Quando la racchetta si trova in seconda colonna e la pallina in prima colonna con direzione SE o in quinta o sesta colonna con direzione SO
\[
3
\]
stati possibili.

Altrettanti stati per simmetria.

Quando la racchetta si trova tra la terza e cinque colonne prima dell'ultima, la pallina nelle due colonne antecedente la racchetta
con direzione SE o nelle duce colonne sucessiva la fine della racchetta
con direzione S0
\[
4 (m-w-4)=4(m-w)-16=24
\]

In totale abbiamo
\[
(1+3)2+4(m-w)-16=4(m-w)-8=32
\]
possibili stati.

\section{Considerazioni}

In tutto gli stati finali sono quindi 
\[
\begin{array}{r}
 2w(m-w)+14(m-w)-34+4(m-w)-8= \\
 =2w(m-w)+18(m-w)-42= \\
 =2(w+9)(m-w)-42= 198\\
\end{array}
\]

Supponiamo ora che la strategia per gli stati precedenti il rimbalzo
porti ad una distribuzione uniforme degli stati precedenti i rimbalzo, abbiamo quindi tre situazioni:

\begin{itemize}
	\item 	
	Rimbalzo certo con probabilità
	 \[ P(rimbalzo)=\frac{2w(m-w)}{2(w+9)(m-w)-42}
	 =\frac{w(m-w)}{(w+9)(m-w)-21}
	 = \frac{10}{33}\approx0.303
	 \]

	\item 	
	Finale certo con probabilità 
	 \[ P(fine)=\frac{14(m-w)-34}{2(w+9)(m-w)-42}
	 =\frac{7(m-w)-17}{(w+9)(m-w)-21}
	 = \frac{53}{99}\approx0.535
	 \]
	 
	 \item 	
	 Rimbalzo condizionato con probabilità 
	 \[ P(condizionale)=\frac{4(m-w)-8}{2(w+9)(m-w)-42}
	 =2\frac{(m-w)-2}{(w+9)(m-w)-21}
	 = \frac{16}{99}\approx0.162
	 \]
\end{itemize}



\end{document}
