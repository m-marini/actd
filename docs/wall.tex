\documentclass[a4paper,11pt]{article}
\usepackage[T1]{fontenc}
\usepackage[utf8]{inputenc}
\usepackage{lmodern}

\title{Neural network notes}
\author{Marco Marini}

\begin{document}

\maketitle
\tableofcontents

\begin{abstract}
Studio del gioco wall
\end{abstract}

\section{Generale}

Wall è un gioco dove una pallina si muove in un campo rettangolare con traiettorie rettilinee diagonali.
I limiti superiore e laterali sono costituiti da muri che
fanno rimbalzare la pallina.
La parte inferiore invece è aperta e una racchetta controllata
da giocatore si muove orizzontalmente permettendo allo stesso di far rimbalzare la pallina all'interno del campo da 
gioco.

$
\begin{array}{ccccccccccccccc}
=	& = & = & = & = & = & = & = & = & = & = & = & = & = & = \\
|	&  &  &  &  &  &  &  &  &  &  &  &  &  & | \\ 
|	&  &  &  &  &  &  &  &  &  &  &  &  &  & | \\ 
|	&  &  &  &  &  &  &  &  &  &  &  &  &  & | \\ 
|	&  &  &  &  &  &  &  &  &  &  &  &  &  & | \\ 
|	&  &  &  &  &  &  &  &  &  &  &  &  &  & | \\ 
|	&  &  &  &  &  & O &  &  &  &  &  &  &  & | \\ 
|	&  &  &  &  &  &  &  &  &  &  &  &  &  & | \\ 
|	&  &  &  &  &  &  &  &  &  &  &  &  &  & | \\ 
|	&  &  &  &  &  &  &  &  &  &  &  &  &  & | \\ 
	&  &  & = & = & = &  &  &  &  &  &  &  &  & 
\end{array} 
$

Definiamo 
\[ n = 10 \] il numero di righe del campo
\[ m = 13 \] il numero di colonne
\[ w = 3 \] la larghezza della racchetta


\section{Spazio degli stati}

In un qualsiasi momento lo stato del gioco è rappresentato dalla posizione
della pallina, la direzione di spostamento della pallina e la posizione della racchetta.
Calcoliamo il numero di stati possibili:

La racchetta può trovarsi in uno degli
\[m - w + 1 = 11 \] possibili posizioni.

Quando la pallina non si trova in prossimtà dei muri o della racchetta
può muoversi in 4 diverse direzioni: NE, SE, SO, NO.
Quindi abbiamo
\[ 4 (n-2)(m-2) (m - w + 1) = 3872 \]
possibili stati della pallina.

Negli angoli superiori la pallina può avere solo una direzione quindi
si aggiungono altri 
\[
	2 (m -w +1) = 22
\] stati.

Quando si trova in prossimità invece del muro superiore o di quelli laterali, la pallina può assumere solo due possibili velocità quindi avremo:
\[ (m -w + 1) 2 [ 2 (n - 2) + m - 2] = 594 \]
ulteriori stati.

Vediamo ora alcuni particolari quando la pallina si trova su nell'angolo
inferiore sx nel qual caso è possibile una sola traiettoria verso l'alto
se la racchetta si trova in prima o seconda posizione (rimbalzo) o verso 
l'uscita del campo negli altri casi, quindi avremo
\[ m-w+1 = 11 \] possibili casi.

Altrettanti se consideriamo quando la pallina si trova nell'angolo inferiore dx.

Nel caso invece la pallina si trovi sulla riga inferiore del campo avremo due possibili traiettorie: NE o  NO se la pallina si trova esattamente sotto la racchetta (rimbalzo) o SE o SO negli altri casi.

Quindi avremo
\[ 2 * (m - 2) = 22 \] stati.

Poi avremo lo stato finale di pallina fuori campo.

In totale quindi possiamo contare
\[
	3872 + 22 + 594 + 11 + 11 + 22 + 1 = 4533
\] possibili stati.


\section{Miglior policy}

La policy migliore è quella di far rimbalzare la pallina una volta che raggiunge la parte inferiore del campo

La pallina rimbalza ogni
\[
	2 (n - 1)
\]
passi.

Il ritorno aspettato allo stato dopo $ i = (1 \dots 2(n-1) $ passi dal rimbalzo sarà quindi
\[
	R_i = \gamma ^ {2(n-1)-i} \sum_{j=0}^{\infty} R^+ \gamma ^ {2(n-1)j} =
	 R^+ \frac{ \gamma ^ {2(n-1)-i}}{1-\gamma ^ {2(n-1)}}
\]


\section{Rimbalzi certi}

Contiamo ora quanti stati che generano premi positivi indipendentemente dalla strategia.

La racchetta si trova completamente a sx, la pallina può trovarsi completamente a sx con solo la direzione NE possibile, oppure si può trovare nelle due colonne successive con direzioni NO o NE, oppure in quarta colonna con direzione NO
\[ 1 + 2(w-1) + 1 = 2w =  6 \]
possibili stati.

Simmetricamente abbiamo altri 6 stati quando la racchetta si trova completamente a dx.

La racchetta si trova tra la seconda colonna e la quartultima colonna, la pallina può trovarsi nelle tre colonne successive con direzioni NO o NE
\[
2w(m-w-2) = 48
\]
possibili stati.

In tutto possiamo contare 
\[
2w+2w(m-w-2) = 2w(m-w-1)=54
\]
possibili stati di rimbalzo certo.


\section{Fine gioco}

Contiamo ora quanti stati dove, indipendentemente dalla strategia, si arriva a fine gioco.

Quando la racchetta si trova in prima colonna e la pallina si trova in quarta o quinta colonna con direzione SE o dalla sesta alla penultima colonna con due direzioni possibili o infine la pallina si trova in ultima colonna direzione SO
\[
2+2(m-w-2)+1]=2(m-w)-1=19
\]
stati possibili.

Quando la racchetta si trova in seconda colonna e la pallina si trova in quinta o sesta colonna con direzione SE oppure tra la settima e la penultima colonna con due direzioni possibili o infine la pallina si trova in ultima colonna con direzione SO
\[
2+2(m-w-3)+1 = 2(m-w)-3 = 17
\]
stati possibili.

Gli stessi stati li abbiamo per simmetria quando la racchetta si trova al lato opposto.

Quando la racchetta si trova tra terza colonna e la sei colonne prima dell'ultima e la pallina si trova nei bordi con singole direzioni, o nelle due adiacenti la sx o la dx della racchetta con singole direzioni o nelle colonne intermedie tra i bordi e le precedenti colonne
\[
(m-w-6)[6 + 2(m-w-6)]=
(m-w-6)[2(m-w)-6]=
2(m-w-6)(m-w-3)=56
\]
stati possibili.

In tutto quindi ci sono
\[
\begin{array}{c}
 2[2(m-w)-1+2(m-w)-3]+2(m-w-6)(m-w-3)= \\
 8(m-w)-8+2(m-w-6)(m-w-3)= \\
 8(m-w-3+3)-8+2(m-w-6)(m-w-3)= \\
 8(m-w-3)+2(m-w-6)(m-w-3)+16= \\
 2(m-w-3)(m-w-2)+16=128 \\
\end{array}
\] stati di fine certa del gioco.

\section{Rimbalzi condizionali}

Contiamo ora gli stati di rimbalzo dipendenti dalla strategia.

Quando la racchetta si trova tra la prima colonna e cinque colonne prima del bordo dx e la pallina in 4 colonne dopo con direzione SO
\[
2(m-w-2)=2(m-w)-4=16
\]
possibili stati e simmetricamente nel verso opposto per un totale di 
\[
4(m-w)-8 = 32
\] stati possibili.

\section{Considerazioni}

In tutto gli stati finali sono quindi 
\[
\begin{array}{c}
 2w(m-w-1)+2(m-w-3)(m-w-2)+16+4(m-w)-8= \\
 2w(m-w-1)+2(m-w-3)(m-w-2)+16+4(m-w)-8= \\
\end{array}
\]

Supponiamo ora che la strategia per gli stati precedenti il rimbalzo
porti ad una distribuzione uniforme degli stati precedenti i rimbalzo, abbiamo quindi tre situazioni:

\begin{itemize}
	\item 	
	Rimbalzo certo con probabilità $ P(rimbalzo)= $

	\item 	
	Finale certo con probabilità $ P(fine) = $
\end{itemize}



\end{document}
